% !TEX root = migratedoc.tex
\chapter*{Introduction}
\myabstract{The program \migrate estimates population size, migration, population splitting parameters using genetic/genomic data.}

For many purposes in biology, we need to know the effective population size of a population and also how well populations interact with other populations. There are essentially two very different approaches to getting such information: a behavioral or ecological approach that monitors individuals in a focus population and recognizes residents and newcomers. Often individuals are marked with tags or other means (banding in birds, toe clipping in amphibians, and more recently inserting magnetic tags under the skin of animals). Such approaches are complex with large populations, or a small number of immigrants, or species with a hidden lifestyle. \\

Since 1960 an alternative approach has been used. This approach uses an individual's genetic makeup as a tag and measures similarities (or differentials) among groups of individuals. This work led to estimators such as $F_{ST}$, which indicate how isolated populations are from each other and several other measures that are based on allele frequencies within populations or individuals. These methods are most often based on simple population models that Sewall Wright and Ronald Fisher invented. The most common applications used the Wright-Fisher population model that assumes that the population does not grow or shrink, that every individual has the same chance to reproduce, and that adults are replaced by their offspring every generation.
Interestingly, this simple model was (and is) amazingly stable. Even applications to species where such a model seems outlandish (Elephants, humans, etc.) allowed considerable insight into the history of populations. Unfortunately, practitioners are still using these methods despite significant advances in population genetic theory. Problematic issues with these allele-frequency approaches primarily stem from the fact that the assumptions of symmetric immigration rates and equal population sizes need to be fulfilled \citep{beerli:2004:EUP}.\\

Recent approaches based on the coalescent \citep{kingman:2000:oc} allow better formulations of an explicit probabilistic model that can handle different immigration rates and different population sizes and additional complications, such as recombination, population splitting.
These programs come in two classes: site frequency-based method and full probabilistic coalescent-based approaches. The site frequency methods are fast with few individuals and usually force the infinite sites model. The computer program \migrate belongs to the class of full coalescent-based probabilistic model.

\migrate in its most simple form can only handle population sizes, immigration rates, and some forms of population splittings; therefore may not be suitable for all datasets. But often, it may help to decide what to do next, despite potential problems with assumption violation \citep{beerli:2009:wid}. \\
This manual describes the program \migrate, its benefits, but also its shortcomings. You will learn in detail about how to use the program and what options are available. This manual is only a start, I suggest that you subscribe to the \texttt{migrate-support@googlegroups.com} and participate in the community that uses \migrate. Tutorials are also available on the \migrate website.