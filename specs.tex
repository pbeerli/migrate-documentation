\chapter{Files in \migrate}

%\subsection*{Short description of all files \migrate\ can use or generate}
I tried to make it simple and redundant, 
so that there are more than one way to set up things. \migrate\ can use very different ways to manipulate the data and as a result many different files are needed or produced. Minimally, you need the data file, its default name is {\sl infile}, and \migrate\ produces two files that contains results: the outfile (ASCII text file) and a PDF output file that contains the same information (well almost, as you see later, the development of the PDF output has low priority because it consumes a lot of time and without enough grant resources I cannot employ a programmer to make it nicer as it is, but i hope that this eventually will change). The program produces both formats because for quick checking of results the ASCII file can be opened on the command line or with any text viewer, whereas the PDF file requires a PDF reader, for example Adobe Acrobat Reader. Both files should contain the same information (except that the PDF contains histograms for the Bayesian inference and still lacks some of the plots for the ML mode).
\vskip 1cm
\section{Input files}
\begin{center}
\begin{tabular}{l l p{7.0cm} l c}
\hline
Filename & Type & Short description & Necessary?  & Name changeable\\
\hline
infile & Input           & {holds you data} & YES & Yes\\
parmfile  & Input       & {holds options} & - & Yes$^{*}$\\
geofile & Input & {holds a (geographic) distance matrix between the populations} & - & Yes\\
sumfile    & Input      & {holds the summary statistic of the sampled  genealogies from an earlier run, to rerun some statistics}& - & Yes\\
\hline
datefile & Input & holds the date (default is years) of the sample. When used then you need also to supply a generation time and and a mutation rate per year in the {\tt parmfile} or the Menu. & -$^{**}$ & Yes\\
\hline
seedfile  & Input       & {holds a random number seed} & - & No\\
distfile & Input      & {holds a genetic distance matrix} & - & No \\
catfile   & Input       & {holds categories for mutation rate variation} & - & No\\
weightfile   & Input     & {holds weights for each site} & - & No\\
\hline
\multicolumn{5}{l}{* Under Unix the parmfile name ca be given as an argument to the program}\\
\multicolumn{5}{l}{** When different sample dates are used then this file is needed}
\end{tabular}
\end{center}
\subsection{Main input files}
\begin{description}
\item{\bf infile} if this file is not present in the current directory 
than the program will ask for a data file, and you can
give the path to it, you need to type the path, which is for Macintosh and Windows users probably rather uncomfortable. In the {\bf menu} or {\bf parmfile} you can specify an other default name for your datafile. 
\item{\bf datefile} When the samples came from different years and you believe that this makes a different specify the date as the time backward from today (for example years before 2007). With this analysis type, you need to specify a mutation rate in the same units as the dates of the samples.
\item{\bf sumfile} The sumfile allows the reuse of a previous maximum likelihood run (see more under sumfile in the output file section), the data type menu needs to be set to "Genealogy".
\item{\bf bayesallfile} The bayesallfile allows the reuse of a previous Bayesian inference run (see more under sumfile in the output file section), the data type menu needs to be set to "Genealogy". [This is not yet in the full production stage use this with caution, foe example make a save copy of the bayesallfile  before you try this!]
\end{description}
\subsection{Optional input files}
\begin{description}
\item{\bf parmfile} can hold specific menu options, this file and the possible options for the menu are explained in detail in section {\bf menu and parmfile}.
\item{\bf seedfile} holds a random number seed, this is just present for compatibility with PHYLIP, the random number seed can be set in various
ways either in the menu or in the parmfile. [this random number seed option should not be used]
\item{\bf geofile} holds the geographic or arbitrary distances between the populations. When this is used then the migration rates are not only scaled by the mutation rate but also by this distance. This allows to detect environmental barriers when we assume that the genetic potential to migrate is the same in all populations; without a barrier the rates should be all the same per distance unit. The format is like a distance file in the PHYLIP package, but you can use the {\tt \#} as a commentary character.\\
\begin{small}
\begin{verbatim}
# Example geofile for 3 populations, 
# the order of the population must be the same as in the data file
# 
  3
Tallahasse0.0 10.0 150.0
St.Marks  10.0 0.0 160.0
Pensacola 150.0 160.0 0.0
\end{verbatim}
\end{small}

\item{\bf distfile} holds distances between all individuals (need to be in the same order as the data file). The distance file has the same format as the PHYLIP distance file format. Use this only if you suspect that \migrate\ does not recover from its own UPGMA start tree. [This option should probably not be used for data analysis.]

\item{\bf catfile} hold the categories. For each locus you must give
the number of categories, and the value of each category and then a string of
category assignments for each site. You can use the {\tt \#} as a commentary character.\\
\begin{small}
\begin{verbatim}
# Example catfile for two loci with 40 and 30 bp each
#
2 1 10  1111111111111111111122222222222222222222
3 1 3 9 111111111122223333333333222222
\end{verbatim}
\end{small}
\item{\bf weightfile}, for each site and locus you need to give a weight, acceptable weights are
integers from 0 - 9 and letters A-Z, A is the weight 10, B 11 and so on, in total
there are 35 possible different weights possible. { You need  a weight string for
each locus}. [if you use this option please let me know]\\
\begin{small}
\begin{verbatim}
# Example weightfile for two loci with 40 and 30 bp each
# 
1101101101101101101101101101101101101101
33F33F22F22F22F22F22FHHHHHHHHH
\end{verbatim}
\end{small}
\end{description}
\newpage
\section{Output files}
\begin{center}
\begin{tabular}{l p{8.0cm} l c}
\hline
Filename & Short description & Name changeable\\
\hline
parmfile     & {holds options, menu can rewrite this file}  & see menu\\
outfile           & {will be created and replace any file with the same name in the same directory}  & Yes\\
outfile.pdf          & contains the same output as outfile and histograms, you need a PDF viewer to read this file  & Yes\\
bayesfile & contains the histogram data of a Bayesian run (the outfile.pdf used these to generate the posterior distributions. & Yes\\
bayesallfile & contains the raw data of a Bayesian run, can be run through TRACER when only a single replicate and a single locus is used. & Yes\\
mighistfile & [Version 3.0] contain the distribution of migration events over time. & Yes\\
skylinefile & [Version 3.0] contains the distribution of the parameter values over time as calculated by using the expected parameter values for a short time intervals. & Yes\\
treefile         & {holds genealogies, this file will be created and will replace any file with the same name in the same directory} & No\\
mathfile        & {holds plot coordinates for the use in a mathematica notebook, this file will be created and will replace any file with the same name in the same directory} & Yes\\
sumfile         & {holds the summary statistic of the sampled  genealogies for further analysis, this file will be created and will replace any file with the same name in the same directory} & Yes\\
logfile  &  {logs the progress information that is displayed onto the screen into a file} & Yes\\
\hline
\end{tabular}
\end{center}
\subsection{Main output files}
Some conbination of the output files are not possible, for example a Bayesian run will not fill values into the sumfile, etc.
\begin{description}
\item{\bf outfile} and {\bf outfile.pdf}  Somewhere you want to read the results, that is it! The name outfile is the default, but can be changed either in the menu or the parmfile. The PDF file contains graphical representation of some of the table and values. Currently, most of the output is represented in the PDF file, when you used the Bayesian inference setting, with Maximum likelihood there are still some options that are not supported in the PDF file (I still lack a programmer to do all this).
\end{description}
\begin{description}
\item{\bf treefile} holds all, only those of the last chain or 
the best tree(s). The likelihood of each tree is given ($\prob(\mathcal D\ |\ \G)$) in a comment. The programs writes trees with migrations using the Newick format with extensions from the Nexus format. Such trees containing migration events can be printed using the program \texttt{Eventtree} (or short \texttt{ET}) (distributed from http://popgen.scs.fsu.edu/et).
Writing trees to a treefile adds
some burden to the program, it will run slower, especially with the option BEST. Parallel runs increase the communication with the master node and therefore may slow down. 

\item{\bf mathfile [ML only]} holds the raw likelihood surface data, if this was requested in the options. The name mathfile is the default, but can be changed in the menu or parmfile (see appendix). This option seems to produce crashes on some parallel runs. Do not use it on Bayesian inference runs either because the data for the mathfile does not get filled in, this may appear at one time because the plot function that fills in the mathfile could in principle show posterior distributions of all immigrations and all ``emigrations" (this means only the emigrants that successfully arrive at the other populations in the study).

\item{\bf sumfile [ML only]} holds the summaries of all genealogies, if this was requested in the parmfile or menu. The name \textbf{sumfile} is the default. This option allows you to reanalyze a previous run for likelihood ratio test or profile likelihoods.

\item{\bf bayesfile [BI only]} holds the posterior histogram data show in the PDF files. You can use other programs like GNUPLOT or the GMT package to recreate the histograms.
\item{\bf bayesallfile [BI only]} holds the raw posterior values for all parameters. I suggest that you use this option! This option reduces the memory footprint by writing all intermediate results to disk and then rereads them for printing the final results. This file can be also used to independently test whether \migrate converged or not using the program \tracer \citep{Rambaut:2007,Drummond:2007}, \migrate uses a simple 1-step Effective sample size (ESS) calculator that may not always be very accurate, although comparison showed that seeing high autocorrelation in \migrate means to see high autocorrelation (small ESS) in \tracer.

\item{\bf mighistfile} holds the histogram over time of the frequency of migration and coalescence events, with simulated data these plots show typically an exponential decay. When there were changes of parameters over time then the data will enforce different patterns, that can be used to discuss the results.

\item{\bf skylinefile [BI only]} holds the averages of the expected parameter values at specific times. These plots are similar to the skyline plot reported in \beast \citep{Drummond:2005:BCI}, although their derivation is an extension of the original skyline plots of \citep{strimmer:2001:EDH}. \migrate allows in principle to report changes of population sizes and migration rates over time and summarizes over multiple loci, but currently this feature needs more testing on the detetion of changes in migration patterns (I am working an a manuscript on  my version of skyline-plots).
\end{description}
