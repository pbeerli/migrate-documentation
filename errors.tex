\chapter{Errors and warnings displayed by {\tt migrate}}
The errors are in no particular ordering, but
I will move more important ones to the beginning of their sections.

\section{Errors}
The program aborts when it encounters one of the following conditions.
Of course there are certainly conditions
I have not thought of. 

\smallerskip
{\bf SEVERE ERROR: ....}
Most often your {\tt infile} contains a problem (e.g. number of sites
does not match the number actual sites given, number of individuals
does not match). If you fail to correct the problem. please contact me.
 
\smallerskip
{\bf ERROR: Datatype is wrong, please use a valid data type!}\\
{\bf ERROR: the program will crash anyway, so I stop now}\\
You probably specified a wrong letter for the data type in the {\tt parmfile}\\
\smallerskip
{\bf ERROR: Wrong datatype, only the types a, m, s, n} 
{\bf ERROR:         (electrophoretic alleles,}\\
{\bf ERROR:          microsatellite data,}\\
{\bf ERROR:          sequence data,}\\
{\bf ERROR:          SNP polymorphism)  are allowed.} \\
You probably specified a wrong letter for the data type in the {\tt menu}\\
\smallerskip
{\bf ERROR: The parmfile contains an error on line XX}\\
There was a wrong entry or even more likely wrong values in the {\tt parmfile} on line xx.\\
\smallerskip
{\bf ERROR: Inconsistency between your Menu/Parmfile and your datafile}\\
Most likely your parmfile assumes there are n subpopulations and 
you assume m subpopulations. Problems with the migration matrix are likely.\\
\smallerskip
{\bf ERROR: There is a conflict between your menu/parmfile}\\
{\bf ERROR: and your datafile: number of populations are not the same}\\
Most likely your parmfile assumes there are n subpopulations and 
you assume m subpopulations. \\
\smallerskip
{\bf ERROR: cannot find seedfile}\\
You specified that the random number is in {\tt seedfile}, 
but the file is not present
in the directory \migrate\ is running.\\
\smallerskip
{\bf ERROR: Failure to read seed method, should be}\\
{\bf ERROR: seed=auto or seed=seedfile or seed=own:value}\\
{\bf ERROR: where value is a positive integer}\\
Either seed specification in {\tt seedfile} or {\tt parmfile} is wrong.\\
\smallerskip
{\bf ERROR: Failure to read start theta method, should be}\\
{\bf ERROR: theta=FST or theta=Own:x.x}\\
{\bf ERROR: or theta=Own:\{x.x, x.x , x.x, .....\}}\\
{\bf ERROR: migration=Own:{migration value}}\\ 
the start parameters are not correctly specified.\\
\smallerskip
{\bf ERROR: Failure to read start migration method}\\
the start parameters are not correctly specified.\\
\smallerskip
{\bf ERROR: Custom migration matrix was completely set to zero?!}\\
the custom migration matrix was not correctly specified.
\section{Warnings}
{\bf WARNING: migration limit (xx) exceeded: yy}\\
{\bf WARNING: results may be underestimating migration rates}\\
{\bf WARNING: for this chain}\\
If this happens only a few times in short chains, don't worry. If it happens
in the last chain or very often, then your migraiton estimates will be most
likely underestimated, but the migration rates between these populations will
be very high, anyway. It means that there is an upper limit of possible migration events on the genealogies, and this is set as a default to 
number\_of\_populations $\times$ 1000.\\
\smallerskip
{\bf WARNING: Migration forced}\\
{\bf WARNING: results may overestimate migration rates}\\
{\bf WARNING: for this chain}\\
Migration rate is essentially 0.0, the program proposes sometimes a migration
event even so the probabilities would foce a coalescence, this heuristic
helps to escape the fatal attraction to 0.0.
If 4Nm is smaller than 0.1 the program will propose randomly every tenth
event a migration event. This genealogy has then still to be accepted.
Hitting this boundary can produce an upwards bias, but it should be
only be recognizable when your populations are barely connected, if at all.\\
\smallerskip
{\bf WARNIN`G: This does look like sequence data}\\
{\bf WARNING: I just read a number of sites=0}\\
{\bf WARNING: If you use the wrong data type, the program will abort}\\
Check your datatype!\\
\smallerskip
{\bf WARNING: --------------------------------------}\\
{\bf WARNING: Target branch problems with time=xx}\\
{\bf WARNING: --------------------------------------}\\
If you encounter this, abort the program, and try to find the error
in the {\tt infile}, but if the data prints conrrectly, 
please contact me. Probably I should declare this 
a severe error and abort.\\
\smallerskip
{\bf WARNING: proposed and new likelihood differ: xx != yy}\\
{\bf WARNING: abort the program and try to find the errors}\\
{\bf WARNING: there could be a wrong datatype, or infile}\\
{\bf WARNING: to check the data you can print it (see menu)}\\
If you have problems to resolve this error (check for errors in infile), 
please contact me and try to give as much information 
as you can (including your dataset).\\
\smallerskip
{\bf WARNING: Inappropiate entry in parmfile: {\it keyword} ignored}\\
The {\it keyword} of a parmfile entry was wrong, often misspelled.\\
\smallerskip
{\bf WARNING: You forgot to add your guess value:}\\
{\bf WARNING: Theta=Own:{pop1,pop2, ...}}\\
{\bf WARNING: or Theta=Own:guess\_pop (same value for all)}\\
You probably specified Theta=Own and forgot to say what values.\\
\smallerskip
{\bf WARNING: You forgot to add your guess value, use either:}\\
{\bf WARNING: migration=FST}\\
{\bf WARNING: or migration=Own:\{guess\_4Nm\} (same value for all)}\\
{\bf WARNING: or migration=Own:\{ - 4Nm21 4Nm31 .... 4Nm12 - 4Nm23 ...\}}\\
You probably specified migration=Own and forgot to say what values.
See the parmfile section, about how to give the migration values.

