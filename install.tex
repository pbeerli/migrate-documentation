% !TEX root = migratedoc.tex
\chapter{Installation}
\subsection{Binaries}
On UNIX system unpack with \texttt{ tar xvfz migrate.[system].tar.gz} or\\ \texttt{ gunzip -c migrate\VERSION .[system].tar.gz | tar xf -}.
This builds a directory \texttt{ migrate-\VERSION}
with a subdirectory \texttt{ examples},
the files \texttt{ README, HISTORY}, and the programs 
\texttt{ migrate} and \texttt{ migrate-n}. 
The program can be moved to a location like \texttt{ /usr/local/bin} 
and the documentation (HTML files are in documentation/migratedoc) to
your HTML directory (e.g. \texttt{ /usr/local/etc/httpd/htdocs}).
On Powermacs or Windows machines double click the archive 
and a folder system similar the UNIX directories above will be created.

\subsection{Source}
The program is known to compile on
every UNIX machine that has a decent ANSI compatible compiler.
And on the following non-UNIX machines:  INTEL (Windows 2000, xt, vista?).

\subsubsection{UNIX (Linux, BSD Unix, MacOSX)}
\begin{enumerate}
\item {\bt{gunzip -c migrate\VERSION .tar.gz $|$ tar xf -} or\\
\bt{tar xfz migrate\VERSION .tar.gz}}
   this creates a directory "migrate-\VERSION " with "src", ``contribution'',  and "examples" in it.
\item \texttt{ cd migrate-\VERSION }
\item \texttt{ ./configure} \\(this scripts checks your system and will report 
functions the program needs, if a function is not, it will report an error,
which I need to know. I assume that your machine has \texttt{ gcc} installed,
but \texttt{ configure} tries to be smart about other compilers: 
on SGI and DEC ALPHA without \texttt{ gcc} it will use the native 
$cc$ compiler with the approriate options. You can force this behavior
with bash shell: CC=cc ./configure, in csh shell: env CC=cc ./configure

\item  \texttt{ make}  \\
 (please report warnings and especially errors)
 This produces an optimized binary for your computer, if your computer has multiple CPUs or core, you can try to compile
 using \texttt{ make thread}, this produces a binary that can use multiple processors for heated runs, this is good but parallel
 runs on such computers use more CPU cycles.
 
 The result should be a binary \texttt{ migrate} in the migrate directory.
If you have a multiprocessor machine that has the POSIX thread library
installed (the configure script searches for libpthread and pthread.h)
try to use \texttt{ make thread}, this will allow to run the heated chains
in parallel and so should speed up the program if you use heating.
\item \texttt{ make install} \\ (this will install the program and man-page into usr/local/bin, /usr/local/man/man1
   ; you need to be root to do this; this step is not necessary)
\end{enumerate}


\chapter{Parallel \migrate}
This text describes how you can improve the performance of \migrate\ 
when you have more than one locus and more than one computer at 
your fingertips. You can parallelize migrate runs 
(1) using a virtual parallel architecture with a message-passing 
interface (MPI) or (2) by hand.
The hand-version works but is cumbersome, 
the MPI-version runs fine on clusters of MacOSX workstations, dedicated clusters of Linux machines, 
AIX parallel machines (Regatta; SP3, SP4).

\section{I. Using the standard Message passing interface (MPI)}

\begin{enumerate}
\item  Secure as many computers for the analysis as you have loci or parameters
    in your dataset. Make sure that all computers can talk to each
    other. Currently my program will only work if they are a flavor of 
    UNIX (e.g. LINUX or MACOSX). Of course, you need an account on all 
    the machines.
  \begin{itemize}
  \item   Download OpenMPI from  http://www.openmpi.org [I use version 1.2.5] (Macintosh computers with the Leopard operating system -- MacOS 10.5 have this already built-in, use \texttt{ fastmigrate-n})
    \item install on all machines (if this is to complicated for you ask a
      sysadmin or other guru to help:\\
      ~~~~~~\texttt{ ./configure }
      There are several options that may or may not be helpful in your environment
     \item prepare a file ``hostsfile" according to the specs in the openmpi distribution, the master node
      needs to be the first machine mentioned.
      my ``hostsfile" looks like this:\\
      \ttfamily{
      ciguri node=2\\
      zork  node=1\\
      nagual node=32
      }	      
\item make sure that you can access all machines [using ssh]
     without the need to specify a password, see man ssh-keygen and man ssh
     if you have firewalls installed on your individual systems then you would need to allow
     the individual machines to open/request "random" ports on the other machines. 
     On MacOSX machines this is a common problem because the machines may run local firewalls.
\item change into the migrate-2.4/src/ directory
      configure and then use "make mpis-pretty" 
\end{itemize}      

\item If your machines have no cross-mounted file system,
    you need to make sure that the program is all
    in the same path e.g. /home/beerli/migrate-test/migrate-n and on EVERY machine.
    
    
\item Compile migrate, you need to follow the instructions in README. Essentially you need to do\\
\texttt{
     ./configure\\
     make mpis
     }
     
     The configure command sets up the Makefile etc. 
     \texttt{make mpis-pretty} compiles for parallel machines.
     
\item Try run the following command from the src/example directory
\begin{itemize}
  \item    \texttt{mpirun  -np 7 --host hostfile ../migrate-n parmfile.testbayes -nomenu}
    [ 6 loci will be analyzed at at the same time,
     the log is not very comprehensive because all 7 processes
     write to the same console, 7 because there is one master-node
     who does only scheduling  and summarizing, 6 worker-nodes
     do the actual tree rearrangements and the likelihood calculations.
     the number you specify has nothing to do with the physical computers,
     OpenMPI can run several nodes on a single CPU but best is to use not more nodes than there are CPU-cores.

    \item send comments how it worked for you and improvements for my 
     [currently] too short and confusing guide.
\end{itemize}
\end{enumerate}

%\section{II. BY HAND (not recommended)} 
%\begin{enumerate}
%\item Secure as many computers for the analysis as you have loci
%    in your dataset.
%
%\item On one machine prepare a directory with
%\begin{itemize}
%\item \migrate\ \\
%    run the program once, and adjust the run parameters using
%    the menu. Use the sumfile option in the (I)nput menu
%    and then save the parmfile with the 
%    (W)rite parmfile option. Then (Q)uit.
%    Edit the created parmfile and check if you can find
%    \texttt{ 
%    write-summary=YES:sumfile-locus\\
%    then change menu=YES to menu=NO
%}
%\item Copy this directory on each machine and name the directories
%    e.g. locus1 locus2 .....
%    If you use Appleshare be careful that you have also 
%    directories for each locus, or make sure that outfiles, and sumfiles
%    have all different names
%     
%\item Prepare the infiles. One for each locus
%    Copy the infiles into the directories.
%\end{itemize}
%
%\item Start \migrate\ on all machines
%
%\item Once all the \migrate\ runs have finished, 
%    copy all sumfiles onto a single machine
%    it would be helpful if this is your fastest 
%    with lots of RAM. Be careful not ot overwrite
%    individual files (the have the same name" sumfile").
% 
%\item Concatenate the sumfiles 
%
%\item The combined sumfile needs hand editing
%    or you can use the PERL script 
%    
%    \texttt{concat-sumfile}
%    
%    if you cannot run the PERL script or 
%    want to do it by hand, see the example below.
%
%\item make a save copy of the fixed combined sumfile
%
%\item run \migrate\ \\
%    and use option (D)atatype and there (g)enealogy
%    and change other menu items if you want.
%    
%\item voil\`a, a multilocus outfile in a fraction of
%     the time the program needs to run on a single machine.
%
%\end{enumerate}
%
%\begin{figure}[bht]
%
%\begin{center}
%
%\begin{boxedminipage}{16cm}
%\begin{small}
%\texttt{
%\begin{verbatim}
%# begin genealogy-summary file of migrate 2.0.3 ------
%#
%1 3 9 0 1
%0 0 ####### locus 0, replicate 0 ################    <<<<<<<<<<<<<<change this
%1 0 0
%                   0                    0                    0
%101 0.01224715726902237366 0.24028906596661075978 68
%0.01797109426997086853 0.46646854779862101381 88
%0.00810026471633800565 0.36704951583234807222 98
%0.010000 0.010000 0.010000 32.000000 23.000000 23.000000 29.000000 21.000000 27.000000 
%3.53366272929252732415e-03 3.53366273324574875492e-03
%5.30077896995009931885e-03 5.30077895638609870171e-03
%3.74540337472894320145e-03 3.74540325539807769997e-03
%2.61285099904684057037e+03 2.61285112767777718545e+03
%1.87798679166376041394e+03 1.87798680725500389599e+03
%1.27983305210177809386e+03 1.27983304107036315145e+03
%1.61370252198881439654e+03 1.61370251775554697815e+03
%2.59250790064689090286e+03 2.59250788575867818508e+03
%3.33322424226003886361e+03 3.33322432150516533511e+03
%39 1.71782306759650010393e-06
%# end genealogy-summary file of migrate 2.0.3 ------
%\end{verbatim}
%\texttt{
%\end{small}
%\end{boxedminipage}
%\end{center}
%\caption{{\sf First few lines of a sumfile}}
%\label{SUMFILE}
%\end{figure}
%\subsection{What to edit in a sumfile} 
%
%\begin{enumerate}
%\item the heading of a sumfile needs the two first comment lines
%\begin{verbatim}
%# begin genealogy-summary file of migrate 0.9.8 ------
%#
%\end{verbatim}
%
%\item the third line needs editing, the first number is the number of loci
%for single locus data it is 1, change it to the number of loci
%\begin{verbatim}
%1 3 9 0 1  [before]
%4 3 9 0 1  [after, 4 loci]
%\end{verbatim}
%
%\item Search for \#\#\#\#\#\#\# you will find  lines like the following
%\begin{verbatim}
%0 0 ####### locus 0, replicate 0 ################
%\end{verbatim}
%
%the file start couting with 0, so the lines reads locus 1 and replicate 1
%leave the first occurrence as it is. Goto the end of the file and 
%remove
%\begin{verbatim}
%# end genealogy-summary file of migrate 0.9.8 ------
%\end{verbatim}
%
%\item Prepare the next sumfile.x to the master sumfile
%    \begin{itemize}
%    \item Remove everything above \texttt{0 0 \#\#\#\#\#\#\#\# locus 0, replicate 0 .....} 
%    \item change the number to    \texttt{1 0 \#\#\#\#\#\#\#\# locus 1, .....}
%      if you use replicates you need to change the replicates accordingly.
%    \item Remove the last line [except for the very last sumfile
%    \end{itemize}
%\item concatenate the above sumfile-fragment to the master sumfile
%
%\item Goto (4) until done 
%\end{enumerate}
%   

\newpage
