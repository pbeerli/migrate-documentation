\section{How to avoid conflicts with other computer users}
The run time {\bf and the memory usage} of {\tt migrate} is 
highly dependent on the number of populations,
the length of the chains, and the number of loci.
It is common that a single locus data set can run for many hours even
very fast machines, resulting in runs of many days for
multilocus data sets.  For some users this can produce a problem,
either the system administrator or other users gets mad about you
consuming ``all'' resources, this is mostly CPU and for large data
sets also memory. 
\par
{\bf For UNIX systems} the immediate, but perhaps wrong, 
answer to this people is that these demanding programs are one 
of the reasons to use these fast computers; 
a run of {\tt migrate} does normally not compromise
any editing, mail reading, word processing on shared machines.
To free a terminal you can put migrate into background and log out.
\begin{enumerate}
\item Run {\tt migrate-n}
\item Change the menu as you think is apropriate.
\item In the main menu use {\bf (W)rite a parmfile}.
\item Kill the program (Control-c) or use {\bf (Q)uit}.
\item Edit the {\tt parmfile} and change the entry {\tt menu=YES}
to {\tt menu=NO} and any other option you want to change.
If you intend to run the program several times you should change
for each run the {\tt random-seed=OWN:somenumber}.
\item Rerun the program with\\
{\tt nohup (nice migrate-n > migrate.log ; date | $\setminus$
mail -s ``migrate finished'' youremailaddress) \&}\\
the {\tt nohup} allows you to logout without stopping the program, 
additionally potential output is logged into nohup.out.
The {\tt nice} causes to program to run slower when other users are 
using the machine ``unniced''. On servers the nicing often happens
automatically after some time or they have a specific batch system, 
ask you system administrator what's
best for a long run.\\
(b) OR: use the \textsl{screen} command: screen ; migrate-n <CNTRL>-a D D.
<CNTRL>-a D D is a key sequence using the control key a lowercase and two capitalized D, this results in an immediate logout, but leaves the program running. 
\item (a) logout or do something else, you will get mail when
migrate has finished, if you are curious and want to known when approximately it will finish peek into the file {\tt migrate.log}, but do not save it.\\(b) On the next login use \textsl{screen -r} to recover the session.
\end{enumerate}
\par
{\bf For  Windows} the program is 
unfortunately not a so good citizen and is disturbing other programs.
To run long {\tt migrate} on these machines the best way is
to run this on a private machines, where you have the control.
\newpage
